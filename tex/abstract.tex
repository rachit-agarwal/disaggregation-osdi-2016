\begin{abstract}
Traditional datacenters are designed as a collection of servers, each of which tightly couples the resources required for computing tasks. Recent industry trends suggest a paradigm shift to a disaggregated datacenter (DDC) architecture containing a pool of resources, each built as a standalone resource blade, and interconnected using a network fabric. 

A key enabling (or blocking) factor for disaggregation will be the network -- to support good application-level performance it becomes critical that the network fabric provide low latency communication even under the increased traffic load that disaggregation introduces. In this paper, we use a workload-driven to derive the minimum latency and bandwidth 
requirements that the network in disaggregated datacenters must provide to avoid degrading application-level performance, and explore the feasibility of meeting these requirements with existing system designs and commodity networking technology.


\eat{In this paper, we use a workload-driven approach to examine the bandwidth and latency requirements for the network fabric in disaggregated datacenters, and to explore whether existing (deployed or proposed) designs meet these requirements.}

\end{abstract}
