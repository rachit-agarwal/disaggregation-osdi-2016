\begin{abstract}
Traditional datacenters are designed as a collection of servers, each of which tightly couples the resources required for computing tasks. Recent industry trends suggest a paradigm shift to a disaggregated datacenter (DDC) architecture containing a pool of resources, each built as a standalone resource blade, and interconnected using a network fabric. In this paper, we use a workload-driven approach to examine the bandwidth and latency requirements for the network fabric in disaggregated datacenters, and to explore whether existing (deployed or proposed) designs meet these requirements.
\end{abstract}
