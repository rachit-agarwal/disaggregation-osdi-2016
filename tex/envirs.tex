
\frenchspacing
\pagestyle{plain}
%\usetikzlibrary{patterns,shapes.geometric,arrows,fit,matrix,positioning,calc}
%\captionsetup[table]{skip=-4pt}
%\Urlmuskip=0mu plus 1mu
\hypersetup{
  colorlinks=true,      % false: boxed links; true: colored links
  linkcolor=blue,       % color of internal links
  citecolor=magenta,    % color of links to bibliography
  filecolor=cyan,       % color of file links
  urlcolor=red          % color of external links
}

% BEGIN Theorem environment settings
%\theoremstyle{plain}
%\theorempreskipamount1pt
%\theorempostskipamount1pt
%\newtheorem{theorem}{Theorem}[section]
%\newtheorem{lemma}[theorem]{Lemma}
%\newtheorem{corollary}[theorem]{Corollary}
%\newtheorem{claim}{Claim}[section]
%\newtheorem{remark}{Remark}
%\newtheorem{example}{Example}[section]
%\newtheorem{conjecture}{Conjecture}[section]
%\newtheorem{defn}{Definition}[section]

%\theorembodyfont{}
%\newtheorem*{myNote}{Note}

\definecolor{boxcolor}{gray}{0.9}
\newenvironment{colframe}{%
  \begin{Sbox}
    \begin{minipage}
      {0.96\columnwidth}
    }{%
    \end{minipage}
  \end{Sbox}
  \begin{center}
    \colorbox{boxcolor}{\TheSbox}
  \end{center}
}


% Proofs get merged into text, with a "Proof" flag and box at the end
%\theoremstyle{nonumberplain}
%\theoremheaderfont{\normalfont\bfseries}
%\newcommand{\proofsubhead}[1]{\noindent{\normalfont\bfseries\boldmath#1}}
%\theoremsymbol{$\blacksquare$} % {~{\rule[0.2ex]{1.17ex}{1.2ex}}}
%\theorembodyfont{}
%\theorempreskipamount0pt
%\theorempostskipamount0pt
%\theoremindent0pt\relax

%\newtheorem{proof}{Proof}
%\newtheorem{proofSketch}{Proof Sketch}
%
%\theoremstyle{break}
%\theoremheaderfont{\bfseries}
%\theorembodyfont{}
%\theoremsymbol{}
% END Theorem environment settings

\newenvironment{denseitemize}{
\begin{itemize}[topsep=2pt, partopsep=0pt, leftmargin=1.5em]
  \setlength{\itemsep}{4pt}
  \setlength{\parskip}{0pt}
  \setlength{\parsep}{0pt}
}{\end{itemize}}

\newenvironment{denseenum}{
\begin{enumerate}[topsep=2pt, partopsep=0pt, leftmargin=*]
  \setlength{\itemsep}{4pt}
  \setlength{\parskip}{0pt}
  \setlength{\parsep}{0pt}
}{\end{enumerate}}

\makeatletter
\def\namedlabel#1#2{\begingroup
   \def\@currentlabel{#2}%
   \label{#1}\endgroup
}
\makeatother

\newcommand{\rc}[1]{{\color{red}{#1}}}
\newcommand{\rqc}[1]{{\color{blue}{#1}}}
\newcommand{\cut}[1]{}
\renewcommand{\ttdefault}{txtt}
\newcommand{\paragraphb}[1]{\vspace{0.075in}\noindent{\bf #1.}}
\newcommand{\paragrapha}[1]{\vspace{0.075in}\noindent{\bf #1}}
\newcommand{\paragraphc}[1]{\vspace{0.075in}\noindent{#1}}
\newcommand{\todo}[1]{\textcolor{Red}{\{#1\}}}

\makeatletter

%\let\OldStatex\Statex
%\renewcommand{\Statex}[1][3]{%
%  \setlength\@tempdima{\algorithmicindent}%
%  \OldStatex\hskip\dimexpr#1\@tempdima\relax}
\makeatother

%\algnewcommand\algorithmicswitch{\textbf{switch}}
%\algnewcommand\algorithmiccase{\textbf{case}}
%\algnewcommand\algorithmicassert{\texttt{assert}}
%\algnewcommand\Assert[1]{\State \algorithmicassert(#1)}%
%
%\algdef{SE}[SWITCH]{Switch}{EndSwitch}[1]{\algorithmicswitch\ #1\ \algorithmicdo}{\algorithmicend\ \algorithmicswitch}%
%\algdef{SE}[CASE]{Case}{EndCase}[1]{\algorithmiccase\ #1}{\algorithmicend\ \algorithmiccase}%
%\algtext*{EndSwitch}%
%\algtext*{EndCase}%

\cut{
\usepackage{draftwatermark}
\SetWatermarkText{DRAFT -- DO NOT REDISTRIBUTE \ \ \ \ \ \ \ DRAFT -- DO NOT REDISTRIBUTE}
\SetWatermarkFontSize{1.5cm}
\SetWatermarkAngle{307}
\SetWatermarkLightness{0.85}
}

% Overline
\newcommand\coline[2]{\colorlet{temp}{.}\color{#1}\overline{\color{temp}#2}\color{temp}}

% TiKZ
%\tikzstyle{cell}=[draw,minimum size=1.5em,text width=1.3em,align=center,anchor=center,font=\small]
%\tikzstyle{ghostcell}=[draw=white,minimum size=1.5em,text width=1.3em,align=center,anchor=center,font=\small]
%\tikzstyle{firstcell}=[draw,minimum size=1.5em,text width=5.5em,align=center,anchor=center,fill=orange!40,font=\small]
%\tikzstyle{firstghostcell}=[draw,minimum size=1.5em,text width=5.5em,align=center,anchor=center,font=\small]
%\tikzstyle{treenode}=[align=center, inner sep=0pt, text centered, text=white, font=\footnotesize]
%\tikzstyle{concat} = [treenode, circle, draw=red!70!white, minimum width=1.2em, fill=red!70!white]
%\tikzstyle{wildcard} = [treenode, circle, draw=green, minimum width=1.2em, fill=green]
%\tikzstyle{union} = [treenode, circle, draw={rgb:red,1;green,2;blue,5}, minimum width=1.2em, fill={rgb:red,1;green,2;blue,5}]
%\tikzstyle{repeat} = [treenode, circle, draw=black, minimum width=1.2em, fill=black]
%\tikzstyle{subtree} = [treenode, regular polygon, regular polygon sides=3, draw=gray, text width=1em, fill=gray, anchor=north,font=\footnotesize]
%\tikzstyle{token} = [align=center, font=\footnotesize]
%\tikzstyle{query} = [align=center, fill=cyan!20, font=\small]
%\tikzstyle{comment} = [align=center]
%\tikzstyle{level} = [sibling distance = 7em/#1, level distance = 2.5em]
%\tikzstyle{loopbelow} = [in=-120, out=-60, loop]

% Table
\newcommand{\specialcell}[2][c]{%
  \begin{tabular}[#1]{@{}c@{}}#2\end{tabular}%
}

